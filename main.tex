% \documentclass[review]{elsarticle}
\documentclass[utf8, babel, sor, jor, amsmath, amssymb, reprint]{elsarticle} %удалить перед отправкой
\usepackage[T2A]{fontenc} %удалить перед отправкой
\usepackage[utf8x]{inputenc} %удалить перед отправкой
\usepackage[english,russian]{babel} %удалить перед отправкой
\graphicspath{{images/}}

\usepackage{lineno,hyperref}
\usepackage{algorithm}
\usepackage{algorithmic}
\modulolinenumbers[5]

\journal{Computer Physics Communications}

\bibliographystyle{elsarticle-num}

\usepackage{mathrsfs}
\usepackage{amsmath}
\usepackage{amssymb}%



\begin{document}
	
	\begin{frontmatter}
		
		
		\title{Exact solution algorithm of $\pm J$ Ising model}
		
		\author[mainaddress, secondaryaddress]{Viacheslav Trukhin\corref{mycorrespondingauthor}}
		\ead{trukhin.vo@dvfu.ru}
		
		\author[mainaddress, secondaryaddress]{Konstantin Nefedev\corref{mycorrespondingauthor}}
		\ead{nefedev.kv@dvfu.ru}

		\author[mainaddress, secondaryaddress]{Eliza Lobanova\corref{mycorrespondingauthor}}
		\ead{lobanova.eal@dvfu.ru}

		\author[mainaddress, secondaryaddress]{Anisich Alexandr\corref{mycorrespondingauthor}}
		\ead{anisich.ai@dvfu.ru}
		
		\address[mainaddress]{Far Eastern Federal University, Vladivostok, Russky Island, 10 Ajax Bay, 690922, the Russian Federation}
		\address[secondaryaddress]{Institute of Applied Mathematics, Far Eastern Branch, Russian Academy of Science, Vladivostok, Radio 7, 690041, the Russian Federation}
		
		\begin{abstract}
			
			
		\end{abstract}
		
		
		\begin{keyword}
			Ising model, GPU and CPU high performance calculations, spin ice, spin glass, statistical thermodynamics.
			
		\end{keyword}
		
		
	\end{frontmatter}
	
	\linenumbers
	\newpage
	\tableofcontents
	
	\newpage
	\section{Введение}

	Задача решения больших графов полным перебором является с одной стороны основной проблемой для математических моделей не приближенное решение которых является критичным для прогнозирования и описания реальных событий и экспериментов, с другой стороны может служить отличной мерой эффективности как аппаратного так и программного обеспечения. Работа по оптимизации таких решений активно ведется \cite{romero2020high}. 
	
	В данной работе представлен алгоритм решения планарного графа, представленного в виде плоской квадратной решетки Изинга с обменными интегралами $\pm 1$ (уточнение Эдвардса-Андерсена). В первой главе представлены подробности этой модели.
	
	Во второй главе рассматривается математический подход к решению, его преимущества и недостатки.
	
	Третья глава посвящена алгоритму.
	\begin{itemize}
		\item GEMC- таблица для каждого столбца
		\item каждая C первого столбца идёт в отдельный поток и создётся GEM-срез новой таблицы по следующим формулам:
		\begin{equation*}
			E_\Sigma = E_r + E_l + E_m,
		\end{equation*}
		\begin{equation*}
			M_\Sigma = M_r + M_l,
		\end{equation*}
		\begin{equation*}
			G[M, E] += G_r
		\end{equation*}
		\item т.к. каждая C представляется двоичным числом, по умолчанию мы ограничены решёткой 8х8 (unsigned long long int = 2**64). Чтобы обойти это ограничение - вносим дополнительную ось в таблице с остатками от деления на 10 ближайших к тысяче простых чисел.
	\end{itemize}
	
	
	\section{Благодарности}
	
	
	
	\bibliography{bibliography.bib}
	
	
\end{document}