\documentclass[10pt]{article}
\usepackage{femj_ru}
\usepackage{algorithm}
\usepackage{algorithmic}

%% Перед отправкой в журнал:
% 1. Перевести файл в кодировку windows-1251
% 2. в файле femj_ru.sty поменять \RequirePackage[utf8]{inputenc} на \RequirePackage[cp1251]{inputenc}
% 3. в файле femj_ru.sty удалить строку 10 \RequirePackage{soulutf8}

%Глубокоуважаемые авторы!

% прежде чем подключать дополнительные пакеты TeXa и создавать новые окружения, пожалуйста, ознакомьтесь с уже предусмотренными в стилевом файле журнала

% Используемые пакеты  перечислены в файле femj_ru.sty строки 7-30
% Обозначения окружений журнала и дополнительные команды приведены в стилевом файле femj_ru.sty строки 38-80

\begin{document}
	
	\Pages(0--0)
	
	\def\Im{\mathop{\mathrm{Im}}\nolimits}
	
	\summary Trukhin~V.\,O.$^{1, 2}$\!, Lobanova~E.\,A.$^{1, 2}$\, Anisich~A.\,I.$^{2}$\author
	
	Application of the Chinese residue theorem for dealing with large numbers in supercomputing\title
	
	The paper presents a parallel algorithm for the exact solution of the \(\pm J\) Ising model. The main focus is on computational methods for analyzing planar graphs modeled on the Ising square lattice with variable exchange integrals. The algorithm is based on a complete enumeration of the system's states and demonstrates high efficiency when using a graphics processing unit (GPU). The authors describe the structure of the algorithm in detail and present the results of its application to problems in statistical thermodynamics. Special attention is given to comparing the execution time of the proposed algorithm with other methods, such as classical brute-force algorithms implemented in Python and C++, as well as using OpenMP and CUDA technologies. The proposed approach allows for precise calculations for systems with arbitrary lattice sizes.
	
	Keywords: algorithms, statistical thermodynamics, Ising model, parallel computing, GPU.
	
	\keywords{Metropolis algorithm, statistical thermodynamics.}
	\org{$^1$ Institute for Applied Mathematics, Far Eastern Branch, Russian Academy of Sciences\\ 
		$^2$ Department of theoretical physics, Far eastern federal university, Russia}
	\references{ %транслитерированный список литературы
		
		\begin{thebibliography}{4}
			\setlength{\parsep}{0pt}\setlength{\itemsep}{3pt}
					
			\bibitem{romero2020high}
			\by Romero, Joshua and Bisson, Mauro and Fatica, Massimiliano and Bernaschi, Massimo
			\jour Computer Physics Communications
			\paper High performance implementations of the 2D Ising model on GPUs
			\vol 256
			\yr 2020
			
			\bibitem{Markovich2019}
			\by Markovich LA
			\jour Information Technology and Systems
			\paper Parallel algorithm based on the Ising model for solving combinatorial optimization problems
			\pages 350--358
			\yr 2019
			
			\bibitem{papadimitriou1977euclidean}
			\by Papadimitriou, Christos H
			\jour Theoretical computer science
			\paper The Euclidean travelling salesman problem is NP-complete
			\vol 4
			\issue 3
			\pages 237--244
			\yr 1977
			
			\bibitem{karp2010reducibility}
			\by Karp, Richard M
			\jour Springer
			\paper Reducibility among combinatorial problems
			\yr 2010
			
			\bibitem{onsager1944crystal}
			\by Onsager, Lars
			\jour Physical Review
			\paper Crystal statistics. I. A two-dimensional model with an order-disorder transition
			\vol 65
			\pages 117
			\yr 1944
			
			\bibitem{zhang2023exact}
			\by ZHANG, Zhidong
			\jour Acta Metall Sin
			\paper Exact Solution of Ferromagnetic Three-Dimensional (3D) Ising Model and Spontaneous Emerge of Time
			\vol 59
			\issue 4
			\pages 489--501
			\yr 2023
			
			\bibitem{maren1991logical}
			\by Maren, Alianna J
			\jour Proceedings of the Second Workshop on Neural Networks
			\paper A logical topology of neural networks
			\yr 1991
			
			\bibitem{grant2020adiabatic}
			\by Grant, Erica K and Humble, Travis S
			\jour Oxford Research Encyclopedia of Physics
			\paper Adiabatic quantum computing and quantum annealing
			\yr 2020
			
			\bibitem{Korol2021}
			\by Korol, Alyona Olegovna and Captain, Vitaly Yurievich
			\jour Far Eastern Mathematical Journal
			\paper Neural network for determining the Curie temperature of the two-dimensional Ising model
			\vol 21
			\issue 1
			\pages 51-60
			\yr 2021
			
			\bibitem{janke2008monte}
			\by Janke, Wolfhard
			\jour Computational many-particle physics
			\paper Monte Carlo methods in classical statistical physics
			\pages 79-140
			\yr 2008
			
			\bibitem{loh2006efficient}
			\by Loh, Yen Lee and Carlson, Erica W
			\jour Physical review letters
			\paper Efficient algorithm for random-bond Ising models in 2d
			\vol 97
			\issue 22
			\pages 227-205
			\yr 2006
			
			\bibitem{katz2007mathematics}
			\by Katz, Victor J
			\jour Physics Letters A
			\paper The Mathematics of Egypt, Mesopotamia, China, India, and Islam: A Sourcebook
			\yr 2007
			
			\bibitem{zhang2018topological}
			\by Zhang, Zhi-Dong
			\jour Many-Body Approaches at Different Scales: A Tribute to Norman H. March on the Occasion of his 90th Birthday
			\paper Topological Effects and Critical Phenomena in the Three-Dimensional (3D) Ising Model
			\pages 331-343
			\yr 2018
			
			\bibitem{zhang2007conjectures}
			\by Zhang, Zhi-Dong
			\jour Philosophical Magazine
			\paper Conjectures on the exact solution of three-dimensional (3D) simple orthorhombic Ising lattices
			\vol 87
			\issue 34
			\pages 5309-5419
			\yr 2007
			
			\bibitem{zhang2013mathematical}
			\by Zhang, Zhi-Dong
			\jour Chinese Physics B
			\paper Mathematical structure of the three-dimensional (3D) Ising model
			\vol 22
			\issue 3
			\yr 2013
			
			\bibitem{zhang2016mathematical}
			\by Zhang, Zhi-Dong
			\jour Acta Metall Sin
			\paper Mathematical structure and the conjectured exact solution of threedimensional (3D) Ising model
			\vol 52
			\issue 10
			\pages 1311--1325
			\yr 2017
			
			\bibitem{zhang2017nature}
			\by Zhang, Zhi-Dong
			\jour Journal of Physics: Conference Series
			\paper The nature of three dimensions: Non-local behavior in the three-dimensional (3D) Ising model
			\vol 827
			\issue 1
			\yr 2017
			
			\bibitem{zhang2020computational}
			\by Zhang, Zhi-Dong
			\jour Journal of Materials Science \& Technology
			\paper Computational complexity of spin-glass three-dimensional (3D) Ising model
			\vol 44
			\issue 5
			\pages 116-120
			\yr 2020
			
			\bibitem{Panchenko2007}
			\by Panchenko, Tatyana Vyacheslavovna and Tarasevich, Yuri Yurievich
			\jour Сomputational methods and programming
			\paper Comparative analysis of the efficiency of application of genetic algorithms and Metropolis algorithm in problems of solid state physics
			\vol 8
			\pages 77-87
			\yr 2007
			
		\end{thebibliography}
	}
	
	\UDC{511.21+517.965+517.547.582}
	\AMS{11B37 + 33E05}
	
	\SupportedBy{Исследование выполнено за счет гранта Российского научного фонда № 24-71-10069, \linebreak https://rscf.ru/project/24-71-10069/\linebreak}
	
	\submitted{ октября 2024 г.}
	
	\title{Параллельный алгоритм точного решения $\pm J$ модели Изинга}
	
	\author[1,2]{В.\, О.~Трухин}{Департамент теоретической физики и интеллектуальных технологий, Институт наукоемких технологий и передовых материалов, Дальневосточный федеральный университет. 690922, Россия, г. Владивосток, о. Русский, п. Аякс, 10}{trukhin.vo@dvfu.ru}
	\author[1,2]{Э.\, А.~Лобанова}{Институт прикладной математики, Дальневосточное отделение Российской академии наук. 690041, Россия, г. Владивосток, Ул. Радио д. 7}{}
	\author[1]{А.\, И.~Анисич}{}{anisich.ai@dvfu.ru}
	\author[1,2]{К.\, В.~Нефедев}{}{}
	
	
	
	
	\makeface
	
	%\markright {Разбавленная модель кубического спинового льда ...} %Добавляем, если длинное название статьи
	%\markleft{В.\, О.~Трухин, В.\, С. Стронгин, Э.\,А.~Лобанова, ...} %добавляем, если много авторов
	
	
	\abstract В статье представлен параллельный алгоритм точного решения \(\pm J\) модели Изинга. Основное внимание уделяется вычислительным методам для анализа планарных графов, моделируемых на основе квадратной решетки Изинга с переменными обменными интегралами. Алгоритм основан на полном переборе состояний системы и демонстрирует высокую эффективность при использовании графического процессора. Подробно описывается структура алгоритма и представляют результаты его применения к решению задач в статистической термодинамике. Особое внимание уделено сравнению времени выполнения предложенного алгоритма с другими методами, такими как классические алгоритмы полного перебора на языках Python и C++, а также с использованием технологий OpenMP и CUDA. Предложенный подход позволяет проводить точные расчеты для систем с произвольным размером решетки.
	
	Ключевые слова: алгоритмы, статистическая термодинамика, модель Изинга, параллельные вычисления, графический процессор.
	
	\keywords{алгоритмы, статистическая термодинамика, модель Изинга, параллельные вычисления, графический процессор.}
	
	\DOI{to be presented}
	
	\section*{Введение}
	
	Задача решения больших графов полным перебором является с одной стороны основной проблемой для математических моделей, решение которых является критичным для прогнозирования и описания реальных событий и экспериментов, с другой стороны может служить отличной мерой эффективности как аппаратного, так и программного обеспечения. Работа по оптимизации таких решений активно ведется \cite{romero2020high}. 
	
	В данной работе представлен алгоритм решения планарного графа, представленного в виде плоской квадратной решетки Изинга с обменными интегралами $\pm 1$ (уточнение Эдвардса-Андерсена~\cite{roma2010ground, katzgraber2005correlation}). 
	
	Для статистической суммы есть точное решение как для двумерной~\cite{onsager1944crystal}, так и для трёхмерной~\cite{zhang2023exact} моделей. Производными этой функции являются такие параметры системы, как свободная энергия и теплоёмкость. Но такие решения не дают возможности моделировать поведение системы Эдвардса-Андерсона в поле или искать её минимум. Для задачи нахождения минимума есть алгоритмы машинного обучения~\cite{maren1991logical} и квантовый алгоритм адиабатического отжига~\cite{grant2020adiabatic}. Нейронные сети~\cite{Korol2021} так же позволяют моделировать систему вблизи фазовых переходов, когда стохастические алгоритмы~\cite{janke2008monte} тратят на термализацию слишком много времени.
	
	Кроме перечисленных выше задач есть задача расчёта свойств системы спинового стекла в поле при наличии разбавлений. Для разбавлений уже есть эффективный алгоритм~\cite{loh2006efficient}, сложность которого, при распараллеливании, совпадает со сложностью нашего алгоритма - $O(L)$, где $L$ - линейный размер системы спинов Изинга.
	
	Однако у точного решения Онзагера~\cite{onsager1944crystal} есть и существенное преимущество - на основе него можно говорить о точности других алгоритмов. Поэтому на рис. 1 мы сравниваем точное решение для энергии и намагниченности с теми графиками, которые даёт наш алгоритм.
	
	\section{Модель Изинга}
	Простое математическое описание модели Изинга (МИ), наряду с отсутствием в настоящее время точного решения, делает ее эталоном в теории сложности, так как любая комбинаторная NP-сложная задача может быть сведена к задаче нахождения минимальной энергии в МИ~\cite{Markovich2019}. Например, она эквивалентна таким комбинаторным задачам оптимизации, как задача коммивояжера (TSP)~\cite{papadimitriou1977euclidean} или задача максимального разреза графа ({}MAX-CUT)~\cite{karp2010reducibility}.
	
	В работе энергия взаимодействия спинов Изинга рассчитывается по формуле
	
	\begin{equation}
		E = \sum\limits_{i=1}^N \sum\limits_{j=1}^N J_{ij} S_i S_j,
	\end{equation}
	где $S_i, S_j = \pm 1$ - значения спинов, $J_{ij} = \pm 1$ - константа обмена между взаимодействующими спинами в модели Эдвардса-Андерсона. Расчёт энергии можно распараллелить, считая по отдельности энергии взаимодействия цепочек спинов, а затем отдельно добавляя взаимодействия между спинами отдельных цепочек. Такой метод распараллеливания удобен и для пересчёта магнитного избытка. 
	
	\section{Математический подход}
	
	Рассмотрим одномерную цепочку из трёх цепочек. Для решения такой цепочки из трёх спинов статистическая сумма принимает вид:
	
	\begin{equation}
		Z_3 = e^{3\beta - 3\beta h} + 3e^{\beta - h - \beta} + 3e^{\beta h - \beta} + e^{3\beta + 3\beta h}
		\label{eq:stat_3}
	\end{equation}
	
	Присоединим ещё одну цепочку параллельно первой:
	
	\begin{equation}
		\label{eq:stat_3_un}
		\begin{alignedat}{2}
			Z_6 = Z_3 e^{\beta  h-\beta }+Z_3 e^{\beta  h-\beta }+Z_3 e^{\beta  h-\beta }+Z_3 e^{3 \beta +3 \beta  h}+ \\
			Z_3 e^{\beta  (-h)-\beta }+Z_3 e^{\beta  (-h)-\beta }+Z_3 e^{\beta  (-h)-\beta }+Z_3 e^{3 \beta -3 \beta  h}
		\end{alignedat}
	\end{equation}
	
	В результате получаем статистическую сумму для решетки $3 \times 2$:
	
	\begin{equation}
		\label{eq:stat_3_res}
		\begin{alignedat}{2}
			Z_6 = 6 e^{-5 \beta }+12 e^{-\beta }+2 e^{3 \beta }+e^{9 \beta -6 \beta  h}+6 e^{3 \beta -4 \beta  h}+6 e^{-3 \beta -2 \beta  h}+\\
			9 e^{\beta -2 \beta  h}+6 e^{2 \beta  h-3 \beta }+9 e^{\beta +2 \beta  h}+6 e^{3 \beta +4 \beta  h}+e^{9 \beta +6 \beta  h}
		\end{alignedat}
	\end{equation}
	
	В таком присоединении и наращивании решетки и заключается идея представленного алгоритма
	
	\section{Алгоритм}
	Далее представлен псевдокод, реализованный в последствии на языке CUDA. По ходу выполнения алгоритма плоская решетка спинов делится на 1D цепи, параллельные друг другу, и постепенно записываются результаты перебора состояний каждой новой цепи. Сначала перебираются все состояния первой цепи и набор параметров GEMC (вырождение-энергия-спиновый избыток-конфигурация спинов в ведущей - последней просчитанной - цепочке) передаётся на $2^L$ потоков. Таким образом, каждый поток перебирает все состояния следующей цепочки только для одного состояния ведущей цепочки). Далее заполняется многомерная матрица G-tensor [конфигурация крайнего слоя][E][M][набор простых числе] с осями под набор конфигураций присоединяемой цепочки, максимальной энергии $\times 2 + 1$, максимального спинового избытка $\times 2 + 1$ и набора простых чисел для записи вырождения, соответственно. В этот тензор в процессе расчёта атомарно добавлялась информация о вырождении по координатам энергии $+ E_{max}$ и спинового избытка $+ M_{max}$.
	
	Поскольку вырождение состояний с одинаковой энергией и спиновым избытком могут принимать значения от $1$ до $\approx 2^{N}$, где $N$ -- размер решетки, то при размерах больше чем $8 \times 8$ вырождения уже не помещаются в целочисленный встроенный тип. Для решения этой проблемы получаемые при решении вырождения разбиваются на простые числа, а после всех расчётов расшифровываются с помощью китайской теоремы об остатках \cite{katz2007mathematics}.
	
	Вычисление GEM для записи одной конфигурации присоединяемой цепи производилось по формулам:
	
	\begin{equation}
		\begin{split}
		E_\Sigma = E_r + E_l + E_m; \\
		M_\Sigma = M_r + M_l; \\
		G[M, E] += G_r,
		\end{split}
	\end{equation}
	
	\noindent где $E_r, E_l, E_m$ - энергии взаимодействий только спинов правой (присоединяемой) цепи между собой, только левой и энергия только смежных взаимодействий соответственно (считается, что цепи добавляются справа налево, аналогичные обозначения приняты для спинового избытка и вырождения). $M_r$ $M_l$ -- спиновые избытки правой (присоединяемой) и левой цепи. $G_r$ -- вырождения состояний правой (присоединяемой) решетки.
	
	Таким образом, каждый из $2^L$ потоков работает над своей конфигурацией ведущей цепи и обновляет общий для всех потоков G-тензор. Алгоритм обеспечивает не одновременную запись для каждой ячейки. 
	
	\begin{algorithm}[H]
		\textbf{ВВОД:} Размер решетки и распределение обменных интегралов.\\
		\textbf{ВЫВОД:} Полная плотность состояний.
		\begin{algorithmic}
			\STATE {Настройка графического процессора:}
			\STATE {Создание массива конфигураций одномерной цепочки методом битового сдвига}
			\STATE {Создание G-тензора [конфигурация крайнего слоя][E][M][набор простых чисел]}
			\FOR {Все слои решетки\\}
			{
				\FOR {Каждая конфигурация внешнего слоя наращиваемой решетки\\}
				{
					\FOR {Каждый добавляемый слой\\}
					{
						\STATE {Расчёт энергии и магнитной восприимчивости}
						\STATE {Атомарное добавление полученного значения вырождения энергии и спинового избытка базовой решетки в тензор вырождения получаемой решетки G[конфигурация крайнего слоя][энергия][спиновый избыток][набор простых чисел]}
					}
					\ENDFOR\\
				}
				\ENDFOR
			}
			\ENDFOR
			\STATE{Расшифровка вырождений набора простых чисел}
			\STATE{Переформирование данных из G-тензора в плотность состояний: вырождение, энергия, магнитная восприимчивость}
		\end{algorithmic}
		\caption{Расчет плотности состояний методом полного перебора.}
		\label{algo:dos_exhaustive}
	\end{algorithm}
	
	В таблице 1 представлено сравнение времени выполнения нашего алгоритма перебора с классическими алгоритмами перебора на других языках.
	
	\begin{table}[h]
		\begin{center}
			\label{Time_Table}
			\caption{Сравнение времён, затрачиваемых на полный перебор систем спинов Изинга с помощью различных программных средств (значения указаны в секундах). В строке "CUDA" представлен полученный алгоритм перебора.}
			\begin{tabular}{|c|c|c|c|c|c|}
				\hline
				& $3 \times 3$ & $4 \times 4$ & $5 \times 5$ & $6 \times 6$ & $7 \times 7$  \\ \hline
				один поток Python & 0            & 4.09         & -            & ---          & ---           \\ \hline
				Python + numba    & 0            & 0.046        & 66           & ---          & ---           \\ \hline
				один поток на C++ & 0.003        & 0.201        & 204.564      & ---          & ---           \\ \hline
				С++ + OpenMP      & -            & -            & 23.443       & ---          & ---           \\ \hline
				CUDA              & 0.305        & 0.602        & 2.385        & 15.153       & 89.114        \\ \hline
			\end{tabular}
		\end{center}
	\end{table}
	
	\section*{Заключение}
	Есть определённая топологическая трудность~\cite{zhang2018topological} в том чтобы использовать обычные низкотемпературные расширения~\cite{katz2007mathematics}, обычные высокотемпературные расширения~\cite{zhang2013mathematical}, пертурбативную группу перенормировки~\cite{zhang2016mathematical} и моделирование Монте-Карло~\cite{zhang2017nature} для моделирования 3D-системы спинов Изинга. Кроме того, уже было показано, что 3D-модель не может иметь алгоритмическую сложность меньше О($2^{mnl}$), где $m$, $n$, $l$ - линейные размеры~\cite{zhang2020computational}. Это делает параллельный подход к точному решению наиболее перспективным в рамках точного моделирования.
	
	Представленный алгоритм выдаёт уникальные характеристики конфигурации системы, поэтому с помощью них можно точно определять, глобальный ли минимум был найдён с помощью, например, алгоритма Монте-Карло~\cite{janke2008monte} или генетического алгоритма~\cite{Panchenko2007}. За счёт него можно вычислить оптимальное количество шагов Метрополиса, необходимых для нахождения глобальных минимумов. Характеристики состояний системы, которые выдаёт алгоритм, позволяют рассчитать свойства спиновых которые без полной плотности состояний получить нельзя. В том числе удалось классифицировать решетку модели Эдвардса-Андерсона в один из видов магнетиков или провести численный эксперимент по определению фазовых свойств таких решеток в поле.
	
	
	\begin{thebibliography}{20}
		\setlength{\parsep}{0pt}\setlength{\itemsep}{3pt}
		
		\bibitem{romero2020high}
		\by Romero, Joshua and Bisson, Mauro and Fatica, Massimiliano and Bernaschi, Massimo
		\jour Computer Physics Communications
		\paper High performance implementations of the 2D Ising model on GPUs
		\vol 256
		\yr 2020
		
		\bibitem{roma2010ground}
		\by Rom{\'a}, F and Risau-Gusman, S and Ramirez-Pastor, AJ and Nieto, F and Vogel, EE
		\jour Physical Review B—Condensed Matter and Materials Physics
		\paper Ground-state topology of the Edwards-Anderson$\pm$J spin glass model
		\vol 82
		\issue 21
		\pages 214-401
		\yr 2010
		
		\bibitem{katzgraber2005correlation}
		\by Katzgraber, Helmut G and Lee, Lik Wee
		\jour Physical Review B—Condensed Matter and Materials Physics
		\paper Correlation length of the two-dimensional Ising spin glass with bimodal interactions
		\vol 71
		\issue 13
		\pages 134-404
		\yr 2005
		
		\bibitem{Markovich2019}
		\by Markovich LA
		\jour Information Technology and Systems
		\paper Parallel algorithm based on the Ising model for solving combinatorial optimization problems
		\pages 350--358
		\yr 2019
		
		\bibitem{papadimitriou1977euclidean}
		\by Papadimitriou, Christos H
		\jour Theoretical computer science
		\paper The Euclidean travelling salesman problem is NP-complete
		\vol 4
		\issue 3
		\pages 237--244
		\yr 1977
		
		\bibitem{karp2010reducibility}
		\by Karp, Richard M
		\jour Springer
		\paper Reducibility among combinatorial problems
		\yr 2010
		
		\bibitem{onsager1944crystal}
		\by Onsager, Lars
		\jour Physical Review
		\paper Crystal statistics. I. A two-dimensional model with an order-disorder transition
		\vol 65
		\pages 117
		\yr 1944
		
		\bibitem{zhang2023exact}
		\by ZHANG, Zhidong
		\jour Acta Metall Sin
		\paper Exact Solution of Ferromagnetic Three-Dimensional (3D) Ising Model and Spontaneous Emerge of Time
		\vol 59
		\issue 4
		\pages 489--501
		\yr 2023
		
		\bibitem{maren1991logical}
		\by Maren, Alianna J
		\jour Proceedings of the Second Workshop on Neural Networks
		\paper A logical topology of neural networks
		\yr 1991
		
		\bibitem{grant2020adiabatic}
		\by Grant, Erica K and Humble, Travis S
		\jour Oxford Research Encyclopedia of Physics
		\paper Adiabatic quantum computing and quantum annealing
		\yr 2020
		
		\bibitem{Korol2021}
		\by Korol, Alyona Olegovna and Captain, Vitaly Yurievich
		\jour Far Eastern Mathematical Journal
		\paper Neural network for determining the Curie temperature of the two-dimensional Ising model
		\vol 21
		\issue 1
		\pages 51-60
		\yr 2021
		
		\bibitem{janke2008monte}
		\by Janke, Wolfhard
		\jour Computational many-particle physics
		\paper Monte Carlo methods in classical statistical physics
		\pages 79-140
		\yr 2008
		
		\bibitem{loh2006efficient}
		\by Loh, Yen Lee and Carlson, Erica W
		\jour Physical review letters
		\paper Efficient algorithm for random-bond Ising models in 2d
		\vol 97
		\issue 22
		\pages 227-205
		\yr 2006
		
		\bibitem{katz2007mathematics}
		\by Katz, Victor J
		\jour Physics Letters A
		\paper The Mathematics of Egypt, Mesopotamia, China, India, and Islam: A Sourcebook
		\yr 2007
		
		\bibitem{zhang2018topological}
		\by Zhang, Zhi-Dong
		\jour Many-Body Approaches at Different Scales: A Tribute to Norman H. March on the Occasion of his 90th Birthday
		\paper Topological Effects and Critical Phenomena in the Three-Dimensional (3D) Ising Model
		\pages 331-343
		\yr 2018
		
		\bibitem{zhang2007conjectures}
		\by Zhang, Zhi-Dong
		\jour Philosophical Magazine
		\paper Conjectures on the exact solution of three-dimensional (3D) simple orthorhombic Ising lattices
		\vol 87
		\issue 34
		\pages 5309-5419
		\yr 2007
		
		\bibitem{zhang2013mathematical}
		\by Zhang, Zhi-Dong
		\jour Chinese Physics B
		\paper Mathematical structure of the three-dimensional (3D) Ising model
		\vol 22
		\issue 3
		\yr 2013
		
		\bibitem{zhang2016mathematical}
		\by Zhang, Zhi-Dong
		\jour Acta Metall Sin
		\paper Mathematical structure and the conjectured exact solution of threedimensional (3D) Ising model
		\vol 52
		\issue 10
		\pages 1311--1325
		\yr 2017
		
		\bibitem{zhang2017nature}
		\by Zhang, Zhi-Dong
		\jour Journal of Physics: Conference Series
		\paper The nature of three dimensions: Non-local behavior in the three-dimensional (3D) Ising model
		\vol 827
		\issue 1
		\yr 2017
		
		\bibitem{zhang2020computational}
		\by Zhang, Zhi-Dong
		\jour Journal of Materials Science \& Technology
		\paper Computational complexity of spin-glass three-dimensional (3D) Ising model
		\vol 44
		\issue 5
		\pages 116-120
		\yr 2020
		
		\bibitem{Panchenko2007}
		\by Panchenko, Tatyana Vyacheslavovna and Tarasevich, Yuri Yurievich
		\jour Сomputational methods and programming
		\paper Comparative analysis of the efficiency of application of genetic algorithms and Metropolis algorithm in problems of solid state physics
		\vol 8
		\pages 77-87
		\yr 2007
		
	\end{thebibliography}
	
	
	
	\EndArticle
\end{document} 