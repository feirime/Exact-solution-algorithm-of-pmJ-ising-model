% \documentclass[review]{elsarticle}
\documentclass[utf8, babel, sor, jor, amsmath, amssymb, reprint]{elsarticle} %удалить перед отправкой
\usepackage[T2A]{fontenc} %удалить перед отправкой
\usepackage[utf8x]{inputenc} %удалить перед отправкой
\usepackage[english,russian]{babel} %удалить перед отправкой
\graphicspath{{images/}}

\usepackage{lineno,hyperref}
\usepackage{algorithm}
\usepackage{algorithmic}
\modulolinenumbers[5]

\journal{Computer Physics Communications}

\bibliographystyle{elsarticle-num}

\usepackage{mathrsfs}
\usepackage{amsmath}
\usepackage{amssymb}%



\begin{document}
	
	\begin{frontmatter}
		
		
		\title{Exact solution algorithm of $\pm J$ Ising model}
		
		\author[mainaddress, secondaryaddress]{Viacheslav Trukhin\corref{mycorrespondingauthor}}
		\ead{trukhin.vo@dvfu.ru}
		
			\author[mainaddress, secondaryaddress]{Elisa Lobanova\corref{mycorrespondingauthor}}
		\ead{lobanova.eal@dvfu.ru}
		
		\author[mainaddress, secondaryaddress]{Alexandr Anisich\corref{mycorrespondingauthor}}
		\ead{anisich.ai@dvfu.ru}
		
		\author[mainaddress, secondaryaddress]{Konstantin Nefedev\corref{mycorrespondingauthor}}
		\ead{nefedev.kv@dvfu.ru}
		
		\address[mainaddress]{Far Eastern Federal University, Vladivostok, Russky Island, 10 Ajax Bay, 690922, the Russian Federation}
		\address[secondaryaddress]{Institute of Applied Mathematics, Far Eastern Branch, Russian Academy of Science, Vladivostok, Radio 7, 690041, the Russian Federation}
		
		\begin{abstract}
			
			
		\end{abstract}
		
		
		\begin{keyword}
			Ising model, GPU and CPU high performance calculations, spin ice, spin glass, statistical thermodynamics.
			
		\end{keyword}
		
		
	\end{frontmatter}
	
	\linenumbers
	\newpage
	\tableofcontents
	
	\newpage
	\section*{Введение}
	
	Задача решения больших графов полным перебором является с одной стороны основной проблемой для математических моделей не приближенное решение которых является критичным для прогнозирования и описания реальных событий и экспериментов, с другой стороны может служить отличной мерой эффективности как аппаратного так и программного обеспечения. Работа по оптимизации таких решений активно ведется \cite{romero2020high}. 
	
	В данной работе представлен алгоритм решения планарного графа, представленного в виде плоской квадратной решетки Изинга с обменными интегралами $\pm 1$ (уточнение Эдвардса-Андерсена). В первой главе представлены подробности этой модели.
	
	Во второй главе рассматривается математический подход к решению, его преимущества и недостатки.
	
	% Третья глава посвящена алгоритму.
	% \begin{itemize}
	% 	\item GEMC- таблица для каждого столбца
	% 	\item каждая C первого столбца идёт в отдельный поток и создётся GEM-срез новой таблицы по следующим формулам:
	% 	\begin{equation*}
	% 		E_\Sigma = E_r + E_l + E_m,
	% 	\end{equation*}
	% 	\begin{equation*}
	% 		M_\Sigma = M_r + M_l,
	% 	\end{equation*}
	% 	\begin{equation*}
	% 		G[M, E] += G_r
	% 	\end{equation*}
	% 	\item т.к. каждая C представляется двоичным числом, по умолчанию мы ограничены решёткой 8х8 (unsigned long long int = 2**64). Чтобы обойти это ограничение - вносим дополнительную ось в таблице с остатками от деления на 10 ближайших к тысяче простых чисел.
	% \end{itemize}
	
	% \section{Сравнение методов расчёта}
	% 
	% Анисич
	% разные методы не только точные их результаты и скорость. Достоинства и недостатки.
	% 
	% \section{Сравнение NP задач}
	% Лобанова
	% 

	\section{Модель Изинга}
	Простое математическое описание модели Изинга (МИ) делает ее эталоном в теории сложности, так как любая комбинаторная NP-сложная задача может быть сведена к задаче нахождения минимальной энергии в МИ~\cite{Markovich2019}. Например, она эквивалентна таким комбинаторным задачам оптимизации, как задача коммивояжера (TSP)~\cite{papadimitriou1977euclidean} или задача максимального разреза графа ({}MAX-CUT)~\cite{karp2010reducibility}.

	В нашей работе энергия взаимодействия спинов Изинга рассчитывается по формуле

	\begin{equation}
		E = \sum\limits_{i=1}^N \sum\limits_{j=1}^N J_{ij} S_i S_j,
	\end{equation}
	где $S_i, S_j = \pm 1$ - значения спинов, $J_{ij} = \pm 1$ - связь между взаимодействующими спинами в модели Эдвардса-Андерсона. Такую энергию легко распараллелить, считая по отдельности энергии взаимодействия цепочек спинов, а затем отдельно добавляя взаимодействия между спинами отдельных цепочек. Такой метод распараллеливания удобен и для пересчёта магнитного избытка. 

	\section{Математический подход}

	Для статистической суммы есть точное решение как для двумерной~\cite{onsager1944crystal}, так и для трёхмерной~\cite{zhang2023exact} моделей. Производными этой функции являются такие параметры системы, как свободная энергия и теплоёмкость. Но такие решения не дают возможности моделировать поведение системы Эдвардса-Андерсона в поле или искать её минимум. Для задачи нахождения минимума есть алгоритмы машинного обучения~\cite{maren1991logical} и квантовый алгоритм адиаббатического отжига~\cite{grant2020adiabatic}. Нейронные сети~\cite{Korol2021} так же позволяют моделировать систему вблизи фазовых переходов, когда стохастические алгоритмы~\cite{janke2008monte} тратят на термолизацию слишком много времени.

	Кроме перечисленных выше задач есть задача моделирования системы спинового стекла в поле или при наличии разбавлений. Для разбавлений уже есть эффективный алгоритм~\cite{loh2006efficient}, сложность которого, при распараллеливании, совпадает со сложностью нашего алгоритма - $О(L)$, где $L$ - линейный размер системы спинов Изинга.

	Однако у точного решения Онзагера~\cite{onsager1944crystal} есть и существенное преимущество - на основе него можно говорить о точности других алгоритмов. Поэтому на рис. 1 мы сравниваем точное решение для энергии и намагниченности с теми графиками, которые даёт наш алгоритм.

	\section{Алгоритм}
	Далее представлен псевдокод, реализованный в последствии на языке CUDA. По ходу выполнения алгоритма система спинов делится на 1D цепи, параллельные друг другу, и постепенно записываются результаты перебора состояний каждой новой цепи. Сначала перебираются все состояния первой цепи и набор параметров GEMC (вырождение-энергия-спиновый избыток-конфигурация спинов в ведущей - последней просчитанной - цепочке) передаётся на $2^L$ потоков. Таким образом каждый перебирает все состояния следующей цепочки только для одного состояния ведущей цепочки). Далее заполняется многомерная матрица G-tensor [config right][E][M][prime number set] с осями под набор конфигураций присоединяемой цепочки, максимальной энергии $\times 2 + 1$, максимального спинового избытка $\times 2 + 1$ и набора простых чисел для записи вырождения, соответственно. В этот тензор в процессе расчёта атомарно добавлялась информация о вырождении по координатам энергии $+ E_{max}$ и спинового избытка $+ M_{max}$.

	Поскольку вырождение состояний с одинаковой энергией и спиновым избытком могут принимать значения от $1$ до $\approx 2^{N}$, где $N$ -- размер решетки, то при размерах больше чем $8 \times 8$ вырождения уже не помещаются в целочисленный встроенный тип. Для решения этой проблемы получаемые при решении вырождения разбиваются на простые числа, а после всех расчётов расшифровываются с помощью китайской теоремы об остатках \cite{katz2007mathematics}.

	Вычисление GEM для записи одной конфигурации присоединяемой цепи производилось по формулам:

	\begin{equation*}
		E_\Sigma = E_r + E_l + E_m,
	\end{equation*}
	\begin{equation*}
		M_\Sigma = M_r + M_l,
	\end{equation*}
	\begin{equation*}
		G[M, E] += G_r,
	\end{equation*}
	где $E_r, E_l, E_m$ - энергии взаимодействий только спинов правой цепи между собой, только левой и энергия только смежных взаимодействий соответственно (считается, что цепи добавляются справа налево, аналогичные обозначения приняты для спинового избытка и вырождения).

	Таким образом, каждый из $2^{11}$ потоков работает над своей конфигурацией ведущей цепи и обновляет общий для всех потоков G-тензор. Алгоритм обеспечивает не одновременную запись для каждой ячейки. 

	\begin{algorithm}[H]
	\textbf{INPUT:} Cell size, exchange integral distribution.\\
	\textbf{OUTPUT:} Full density of states.
	\begin{algorithmic}
		\STATE {GPU configuration:}
		\STATE {creating massive of configurations 1D chain by bit offset}
		\STATE {creating G-tensor [config right][E][M][prime number set]}
		\FOR {Number of layer in cell\\}
		{
			\FOR {Each configuration of the outermost layer of the build-up lattice\\}
			{
				\FOR {Each added 1D chain\\}
				{
					\FOR{Each configuration of added 1D chain\\}
					\STATE {Calculate energy and magnetic susceptibility}
					\STATE {Atomic add degeneration of the build-up lattice in G-tensor[configuration of outermost layer][energy][magnetic susceptibility][prime numer coefficients]}
					\ENDFOR\\
				}
				\ENDFOR\\
			}
			\ENDFOR
		}
		\ENDFOR
		\STATE{Deciphering degeneracies from the set of prime numbers}
		\STATE{Reformating data from G-tensor to degeneracy, energy, magnetic susceptibility density of states}
	\end{algorithmic}
	\caption{Calculation density of states by exhaustive search.}
	\label{algo:dos_exhaustive}
	\end{algorithm}

	В таблице 1 представлено сравнение времени выполнения нашего алгоритма перебора с классическими алгоритмами перебора на других языках.

	\begin{table}[h]
    \begin{center}
        \label{Time_Table}
        \caption{Сравнение времён, затрачиваемых на полный перебор систем спинов Изинга с помощью различных программных средств (значения указаны в секундах). Словом "CUDA" обозначен наш алгоритм перебора.}
        \begin{tabular}{|c|c|c|c|c|c|}
            \hline
                  & $3 \times 3$ & $4 \times 4$ & $5 \times 5$ & $6 \times 6$ & $7 \times 7$  \\ \hline
один поток Python & 0            & 4.09         & ?            & ---          & ---           \\ \hline
Python + numba    & 0            & 0.046        & 66           & ---          & ---           \\ \hline
один поток на C++ & 0.003        & 0.201        & 204.564      & ---          & ---           \\ \hline
c++ + OpenMP      & ?            & ?            & 23.443       & ---          & ---           \\ \hline
CUDA              & 0.305        & 0.602        & 2.385        & 15.153       & 89.114        \\ \hline
        \end{tabular}
    \end{center}
\end{table}

	\section*{Заключение}
	Есть определённая топологическая трудность~\cite{zhang2018topological} в том чтобы использовать обычные низкотемпературные расширения~\cite{katz2007mathematics}, обычные высокотемпературные расширения~\cite{zhang2013mathematical}, пертурбативную группу перенормировки~\cite{zhang2016mathematical} и моделирование Монте-Карло~\cite{zhang2017nature} для моделирования 3D-системы спинов Изинга. Кроме того, уже было показано, что 3D-модель не может иметь алгоритмическую сложность меньше О($2^{mnl}$), где $m$, $n$, $l$ - линейные размеры~\cite{zhang2020computational}. Это делает параллельный подход к точному решению наиболее перспективным в рамках точного моделирования.

	Наш алгоритм выдаёт уникальные характеристики конфигурации системы, поэтому с помощью них можно точно определять, глобальный ли минимум был найдён с помощью, например, алгоритма Монте-Карло~\cite{janke2008monte} или генетического алгоритма~\cite{Panchenko2007}. За счёт него можно вычислить оптимальное количество шагов Метрополиса, необходимых для нахождения глобальных минимумов. На основе характеристик состояний системы, которые выдаёт наш алгоритм, можно классифицировать систему Эдвардса-Андерсона в один из видов магнетиков или смоделировать её поведение в поле.

	
	\section*{Благодарности}
	
	
	
	\bibliography{bibliography.bib}
	
	
\end{document}